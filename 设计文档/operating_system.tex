\chapter{操作系统部分}
本组计划在板上系统上运行 uCore 操作系统,基本规划为:对SOC、操作系统进行调试使 uCore-thumips 系统运行正常,针对平台进行 MMU、外设等方面等相关实现和测试;对 uCore Plus 操作系统进行扩展,完善其对 MIPS 32 平台的支持,同时达到在板上正常运行的目标。

\section{uCore-thumips操作系统的移植与运行}
\label{section:ucore-thumips}
\subsection{系统概述}

uCore-thumips\footnote{\url{https://github.com/z4yx/ucore-thumips}} 是针对简化后的 MIPS 32 实现:MIPS32S 平台的 uCore 移植版本。该项目针对 MIPS32S 平台实现了对应的Bootloader、初始化流程、异常处理、内存管理和上下文切换流程。相比标准的 MIPS 32,MIPS32S 缺少部分指令且不支持延迟槽。针对这些不同,uCore-thumips 对 uCore 操作系统的编译选项进行了相应的修改,并提供了额外的库函数实现缺失的指令(如 \texttt{divu})的功能。

\subsection{编译方法}
在非 mipsel 平台编译、调试 uCore-thumips 需要使用面向 mipsel 架构的交叉编译、调试工具链,所需工具主要包括 binutils、gcc和gdb。

Debian 系统下,\texttt{gcc-mipsel-linux-gnu}和\texttt{binutils-mipsel-linux-gnu}软件包分别提供了预编译的目标平台为 mipsel 的 binutils 和 gcc。其它操作系统的工具链可参考 LinuxMIPS 项目文档\footnote{\url{https://www.linux-mips.org/wiki/Toolchains}}自行编译。此外 Sourcery CodeBench Lite\footnote{\url{https://sourcery.mentor.com/GNUToolchain/release2189}} 提供了预编译的 mipsel 工具链。 

交叉编译时,指定 \texttt{CROSS\_COMPILE} 环境变量或修改 Makefile 中 \texttt{CROSS\_COMPILE} 变量为所使用的交叉编译器,即可使用 make 进行编译。编译后得到镜像 \texttt{ucore-kernel-initrd} 和 \texttt{boot/loader.bin} 分别为系统内核 ELF 和Bootloader。

进行移植时,需针对片上系统对 Makefile 中相应配置进行修改,包括延迟槽、浮点模块等编译选项、为用户 App 预留存储大小等。

\subsection{启动流程}
uCore-thumips 的引导、启动流程主要分为Bootloader加载系统、初始化C环境、初始化系统三个步骤。

uCore-thumips 提供了简易的Bootloader \texttt{boot/bootasm.S},该程序从 Flash(默认为地址0xBE000000)读取合法的ELF文件头,将其复制到内存的相应位置并跳转。

Bootloader加载系统后将跳转至 \texttt{kern/init/entry.S} 中的 \texttt{kernel\_entry} 过程。在此过程中,系统将重置 CP0 中异常相关寄存器、设置 TLB 相关异常向量;同时,正确设置 \texttt{sp, gp},清空\texttt{bss}以满足C程序运行要求,之后跳转至 \texttt{kern/init/init.c} 中的 \texttt{kern\_init} 函数。\texttt{kern\_init} 函数将完成中断控制、控制台、异常、内存管理、进程管理等系统功能的初始化。

进行移植时,需将Bootloader替换为针对TrivialMIPS片上系统自行实现的TrivialBootloader或U-Boot,针对平台对中断控制、控制台等功能的初始化过程进行相应修改。

\subsection{内存管理}

MIPS32 使用软件进行 TLB 缺失处理,当发生 TLB 缺失时会触发 TLB Refill 异常。uCore-thumips 已经实现了 TLB Refill 异常的处理。发生 TLB 缺失时,系统会首先检查页表判断是否为缺页,若为缺页调用 \texttt{do\_pgfault} 进行处理,否则检查权限后填充 TLB 表项。

\subsection{异常处理}
异常处理程序通过访问 CP0 中的 Cause 寄存器获取异常信息,同时需要正确设置 Status 寄存器中的某些位。用户态和特权态切换时,uCore 内核使用 \texttt{trapframe} 结构存储程序运行状况。uCore-thumips 已实现和 CP0 中寄存器的交互及\texttt{trapframe}的保存。

\section{uCore Plus 操作系统的移植与运行}
\subsection{系统概述}
uCore Plus\footnote{\url{https://github.com/oscourse-tsinghua/ucore_plus}} 是 uCore 的全功能版本,且提供 i386、AMD64、ARM、MIPS 等多平台的支持。uCore Plus 被划分为不同模块,各模块可独立编译,通过平台相关的 Makefile 进行组合。为提供多平台支持, uCore Plus 的多数模块都分为架构相关部分和架构无关部分,架构相关部分代码位于各模块\texttt{arch/\$ARCH}目录下。每个模块的架构相关代码都至少包含单独的 Makefile 和链接脚本。

\subsection{移植评估}

\subsubsection{Bootloader}
uCore Plus 的 Bootloader 模块中,架构无关代码只包括 ELF 头和一些通用结构的定义,各架构需实现对应的 Bootloader。uCore Plus 已有的 MIPS Bootloader 沿用了 uCore-thumips 的 Bootloader 的代码。移植时,可以使用TrivialBootloader或U-Boot。

\subsubsection{内核}
\begin{itemize}
\item{驱动程序}
uCore Plus 中,不同架构不共用驱动程序代码,需要为每种架构都实现整套的驱动程序。必须要实现驱动的设备包括console输出、输入、计时器和外部存储。移植时需实现TrivialMIPS平台对应console、计时器和外部存储的驱动。各类设备驱动所需实现的接口见表 \ref{table:console_driver_interface}, \ref{table:timer_driver_interface}, \ref{table:ide_driver_interface}。

\begin{table}[!htbp]
\centering
\begin{tabular}{|c|c|}
\hline
\textbf{接口}                                      & \textbf{说明}                      \\ \hline
\texttt{void cons\_init(void)}  & 初始化 console                      \\ \hline
\texttt{void cons\_putc(int c)} & 输出字符 \texttt{c} \\ \hline
\texttt{int cons\_getc(void)}   & 输入字符                             \\ \hline
\end{tabular}
    \caption{console驱动需实现的接口}
    \label{table:console_driver_interface}
\end{table}


\begin{table}[!htbp]
\centering
\begin{tabular}{|c|c|}
\hline
\textbf{接口}                                      & \textbf{说明}                      \\ \hline
\texttt{void clock\_init(void)}  & 初始化计时器                      \\ \hline
\texttt{size\_t ticks}   & uCore 启动后经过的时间                             \\ \hline
\end{tabular}
    \caption{计时器驱动需实现的接口}
    \label{table:timer_driver_interface}
\end{table}

\begin{table}[!htbp]
\centering
\begin{tabular}{|c|c|}
\hline
\textbf{接口}                                      & \textbf{说明}                      \\ \hline
\texttt{void ide\_init(void)}  & 初始化外部存储                      \\ \hline
\texttt{bool ide\_device\_valid(unsigned short ideno)}   & 检查外部存储是否存在且可用                             \\ \hline
\texttt{size\_t ide\_device\_size(unsigned short ideno)}   & 返回外部存储每个扇区的大小                             \\ \hline
\texttt{int ide\_read\_secs(...)}   & 读取给定扇区数的数据                             \\ \hline
\texttt{int ide\_write\_secs(...)}   & 写入给定扇区数的数据                             \\ \hline
\end{tabular}
    \caption{外部存储驱动需实现的接口}
    \label{table:ide_driver_interface}
\end{table}

\item{初始化}
uCore Plus 已包含 uCore-thumips 中的初始化代码,初始化过程与 uCore-thumips 相同。
\item{内存管理}
uCore Plus 中,至少需为每个架构在 \texttt{pmm.c}中实现 MMU 初始化和 TLB 缺失的处理、在 \texttt{vmm.c} 中实现 \texttt{copy\_from\_user, copy\_to\_user} 和 \texttt{copy\_string} 三个用于在内核态和用户态之间拷贝数据的函数。uCore Plus 已包含来自 uCore-thumips 中的页表实现和TLB管理代码。
\item{同步、系统调用及进程管理} uCore Plus 中已实现 MIPS 平台同步、系统调用及进程管理的平台相关代码,部分来自 uCore-thumips,可参考\ref{section:ucore-thumips}节。移植时不需要太大改动。

\end{itemize}
\subsubsection{ulib}

ulib 是一个静态链接到所有用户程序的函数库,源码位于 \texttt{src/libs-user-ucore/}。其中的部分函数平台相关,移植时需要实现。相关接口及在 MIPS 平台的实现情况见表 \ref{table:ulib_interface}。

\begin{table}[!htbp]
\centering
\begin{tabular}{|c|c|c|}
\hline
\textbf{接口} & \textbf{说明} & \textbf{状态} \\ \hline
\texttt{initcode.S} & 程序入口 & 已实现 \\ \hline
\texttt{clone.S} & 克隆的进程 & 待实现 \\ \hline
\texttt{atomic.h} & 用于用户态锁的实现 & 已实现 \\ \hline
\texttt{syscall.c} & 进行系统调用 & 已实现 \\ \hline
\end{tabular}
    \caption{ulib需实现的接口}
    \label{table:ulib_interface}
\end{table}