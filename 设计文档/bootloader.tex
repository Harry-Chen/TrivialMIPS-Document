\chapter{引导程序(Bootloader)}

\section{第一阶段:TrivialBootloader}

作为板上系统的一部分,Bootrom中包含了系统每次上电或重置时都会首先执行的代码,起始物理地址为 \texttt{0x1FC00000}。由于这部分程序是固化在FPGA中的,为了节约有限的板载存储,Bootrom中的代码不能太大(暂定为16KB)。因此,有必要撰写一个较小的Bootloader来进行初步的系统初始化和加载工作,将其命名为TrivialBootloader。

\subsection{汇编部分}

作为程序入口,在跳转到高级语言编写的代码之前,需要先使用汇编语言设置一些基本参数,如栈基址(\texttt{sp})寄存器和异常处理入口等。具体地,汇编部分具有如下功能:

\begin{itemize}
    \item 系统的全局初始化,设置\texttt{sp, gp}寄存器,跳转到实际代码入口
    \item C++代码退出后的清理与提示
    \item 启用异常处理并设置异常向量
\end{itemize}

其中实际代码入口为 \texttt{\_main} 符号,\texttt{sp, gp}寄存器被分别设置到\texttt{\_stack, \_gp} 符号。而异常处理会跳转到 \texttt{\_exception\_handler} 符号。这些符号需要在链接时被填充。

\subsection{高级语言部分}

高级语言部分也分为两个C与C++两个子部分。C编写的部分主要用于提供 \texttt{\_main} 函数和 \texttt{\_exception\_handler}函数。前者具有打印欢迎信息、清零 BSS 段、执行 \texttt{main} 函数、打印返回值等功能,后者负责处理异常,从CP0中收集异常原因、异常地址等信息,以友好的方式打印出来。

C++部分是Bootloader的主体,也也是真正的 \texttt{main} 函数实现位置,其具有的功能为:

\begin{description}
    \item[内存检测] 通过不同块大小随机读写检测内存硬件与实现是否存在问题
    \item[ELF启动] 支持从非易失存储(Flash)中读取合法的ELF文件头,正确地将其复制到内存的相应位置并跳转
    \item[直接启动] 支持直接跳转到内存入口点(\texttt{0x8000000})启动,便于系统移植时的调试工作
    \item[串口旁加载] 支持从串口直接向内存中加载数据和指令,并跳转到指定的位置启动
    \item[内存转储] 支持将指定的内存区域的数据转储到串口输出
    \item[异常处理] 正确处理各种操作异常、非法情况(如没有选择启动模式、内存检测失败、要复制到内存的数据覆盖了Bootloader本身的代码),在发生异常时通过串口、LED等多种途径给出友好可读的提示
\end{description}

作为功能举例,在引导Flash中操作系统的 ELF 文件时,Bootloader输出为(没有启用内存检查):
\begin{minted}{text}
*****TrivialMIPS Bare Metal System*****
=====Entering TrivialBootloader=====
Bootloader used memory: from 0x80700000 to 0x80800000
Mode: Boot
Device: SPI Flash
Valid ELF file found, will now copy to RAM.
Copying 24 bytes from offset 0x19480 to address 0x80019400
Copying 2159744 bytes from offset 0x80 to address 0x80000000
Booting from address 0x80000000...
=====Exiting TrivialBootloader=====
\end{minted}

\subsection{链接脚本}

由于Bootloader可能被放置在不同位置,此时其本身可用的内存空间(作为栈)以及加载基址是不同的。因此,我们编写了如下的链接脚本。可以看到,我们将内存划分为高低两片,根据代码生成的位置选择不同的区域放置栈代码段和数据段。此外,可用内存范围、BSS段范围、需要检查的内存范围等都是由链接器提供的符号。其可以通过编译时定义不同的常量进行预处理,从而为高级语言提供关于平台的信息,灵活地适应多种要求。

\begin{minted}{c}
ENTRY(_start);

MEMORY
{
    BOOTROM  (rx)  : ORIGIN = 0xBFC00000, LENGTH = 16K
    RAM_BASE (rwx) : ORIGIN = 0x80000000, LENGTH = 7M
    RAM_HIGH (rwx) : ORIGIN = 0x80700000, LENGTH = 1M
}

#ifdef CODE_INTO_BOOTROM
#define TEXT_AREA BOOTROM
#define DATA_AREA RAM_HIGH
#else
#define TEXT_AREA RAM_BASE
#define DATA_AREA RAM_BASE
#endif

SECTIONS
{
    . = ORIGIN(TEXT_AREA);

    _mem_start = ORIGIN(DATA_AREA);
    _mem_end = ORIGIN(DATA_AREA) + LENGTH(DATA_AREA);

#ifdef CODE_INTO_BOOTROM
    _mem_check_start = ORIGIN(RAM_BASE);
    _mem_check_end = ORIGIN(RAM_BASE) + LENGTH(RAM_BASE);
#else
    _mem_check_start = _stack;
    _mem_check_end = ORIGIN(RAM_HIGH) + LENGTH(RAM_HIGH);
#endif

    .text :
    {
        _text = .;
        *(.text.startup)
        *(.text*)
        *(.rodata*)
        *(.reginfo)
        *(.init)
        *(.stub)
        *(.gnu.warning)
        *(.MIPS.abiflags)
        _text_end = .;
    } > TEXT_AREA

    .data :
    {
        _data = .;
        _stack = _data + MEM_SIZE - 32;
        *(.data)
        *(.data*)
        *(.eh_frame)
        _gp = ALIGN(16) + 0x7ff0;  
        *(.got.plt) *(.got)
        *(.sdata) 
        *(.lit8) 
        *(.lit4)
        _data_end = .;
    } > DATA_AREA

    .sbss : 
    { 
        *(.sbss)
        *(.scommon)
    } > DATA_AREA
    
    .bss :
    {
        _bss = .;
        *(.dynbss)
        *(.bss)
        *(COMMON)
        _bss_end = .;
    } > DATA_AREA

    /DISCARD/ :
    {
        *(.note*);
        *(.iplt*);
        *(.igot*);
        *(.rel*);
        *(.comment);
    }
}

\end{minted}

\section{第二阶段:U-Boot}

\subsection{背景}
U-Boot是一个启动引导程序,常见于嵌入式系统中,用于引导Linux等操作系统。通过运行U-Boot引导程序,可以支持从Flash、U盘、网络等来源加载uCore、Linux系统镜像到内存并进行引导。由于U-Boot本身有较强的命令行功能和交互能力,它也可以作为一个硬件测试与演示的工具。在本系统的设计中,U-Boot将作为二级引导程序,放置在Flash中,被TrivialBootloader所加载。

\subsection{硬件需求}

U-Boot对CPU的功能要求较低,它不使用MIPS的中断和TLB机制,因此硬件可以不需要实现这些机制。对于其它的异常,仅仅在程序运行不正常时才会发生,如果假定程序能正常运行,对异常处理也没有要求。

作为功能丰富的引导程序,其对外设的要求很高,将用到表\ref{table:address_allocation}中列出的除了图形设备之外的所有外设(其中定时器用于运行 Dhrystone 等性能测试时的准确时间评估)。其中,USB与以太网模块的正常工作是至关重要的,否则片上系统将失去从外部加载系统的功能。

\subsection{编译方法}

U-Boot可以直接使用\ref{section:software_platform}节中给出的编译器套件进行编译和调试。具体地,如果完成移植,并将配置文件保存为 \texttt{trivialmips\_thinpad\_defconfig},则只需要运行下列命令:

\begin{itemize}
    \item \texttt{make\ CROSS\_COMPILE=mipsel-linux-gnu- trivialmips\_thinpad\_defconfig}
    \item \texttt{make CROSS\_COMPILE=mipsel-linux-gnu- }
\end{itemize}

就能生成最终的 ELF 可执行文件,将其写入 Flash 开头位置,即可被 TrivialBootloader 所引导。

\subsection{移植内容}

U-Boot 与 Linux 内核源码的组织架构类似,也都采用了\texttt{DTS}(设备树源码)来描述设备,因此移植方法也比较类似。主要的移植工作主要分为两部分,一部分是添加CPU相关的SoC支持,一部分是添加板级支持。

由于之前的类似项目已经有了完成度较高的工作\footnote{\url{https://github.com/z4yx/u-boot-naivemips/}},本项目在其基础上进一步适配。主要的工作有:

\begin{itemize}
    \item 添加串口控制器驱动(包括系统启动早期和后期两个部分)
    \item 添加定时器驱动,准确反映运行时间
    \item 修改设备描述以准确反映SoC片上资源,修复驱动错误(包括SL811 USB模块和DM9000A 网卡驱动)
\end{itemize}

\textcolor{red}{Pero TODO: 一些移植的细节(如改动的文件,环境变量,启动菜单等)}