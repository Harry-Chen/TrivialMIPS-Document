\chapter{Bootloader部分}

\section{第一阶段:TrivialBootloader}

作为板上系统的一部分,Bootrom中包含了系统每次上电或重置时都会首先执行的代码,起始物理地址为 \texttt{0x1FC00000}。由于这部分程序是固化在FPGA中的,为了节约有限的板载存储,Bootrom中的代码不能太大(暂定为16KB)。因此,有必要撰写一个较小的Bootloader来进行初步的系统初始化和加载工作,将其命名为TrivialBootloader。

本项目计划用C++与汇编语言实现bootloader。其中汇编部分需要有下列功能:

\begin{itemize}
    \item 系统的全局初始化,设置\texttt{sp, gp}寄存器,跳转到C++代码入口
    \item C++代码退出后的清理与提示
    \item 启用异常处理并设置异常向量
\end{itemize}
C++部分是Bootloader的主体,其应当具有的功能为:

\begin{description}
    \item[内存检测] 通过不同块大小随机读写检测内存硬件与实现是否存在问题
    \item[ELF启动] 支持从非易失存储(Flash)中读取合法的ELF文件头,正确地将其复制到内存的相应位置并跳转
    \item[直接启动] 支持直接跳转到内存入口点(\texttt{0x8000000})启动,便于系统移植时的调试工作
    \item[串口旁加载] 支持从串口直接向内存中加载数据和指令,并跳转到指定的位置启动
    \item[内存转储] 支持将指定的内存区域的数据转储到串口输出
    \item[异常处理] 正确处理各种操作异常、非法情况(如没有选择启动模式、内存检测失败、要复制到内存的数据覆盖了Bootloader本身的代码),在发生异常时通过串口、LED等多种途径给出友好可读的提示
\end{description}

\section{第二阶段:U-Boot}

\subsection{背景}
U-Boot是一个启动引导程序,常见于嵌入式系统中,用于引导Linux等操作系统。通过运行U-Boot引导程序,可以支持从Flash、U盘、网络等来源加载uCore、Linux系统镜像到内存并进行引导。由于U-Boot本身有较强的命令行功能和交互能力,它也可以作为一个硬件测试与演示的工具。在本系统的设计中,U-Boot将作为二级引导程序,放置在Flash中,被TrivialBootloader所加载。

\subsection{硬件需求}

U-Boot对CPU的功能要求较低,它不使用MIPS的中断和TLB机制,因此硬件可以不需要实现这些机制。对于其它的异常,仅仅在程序运行不正常时才会发生,如果假定程序能正常运行,对异常处理也没有要求。

作为功能丰富的引导程序,其对外设的要求很高,将用到表\ref{table:address_allocation}中列出的除了图形设备之外的所有外设(其中定时器用于运行 Dhrystone 等性能测试时的准确时间评估)。其中,USB与以太网模块的正常工作是至关重要的,否则片上系统将失去从外部加载系统的功能。

\subsection{编译方法}

U-Boot可以直接使用\ref{section:software_platform}节中给出的编译器套件进行编译和调试。具体地,只需要运行下列命令:

\begin{itemize}
    \item \texttt{make\ CROSS\_COMPILE=mipsel-linux-gnu- trivialmips\_thinpad\_defconfig}
    \item \texttt{make CROSS\_COMPILE=mipsel-linux-gnu- }
\end{itemize}

\subsection{移植目标}

U-Boot 与 Linux 内核源码的组织架构类似,也都采用了\texttt{DTS}(设备树源码)来描述设备,因此移植方法也比较类似。主要的移植工作主要分为两部分,一部分是添加CPU相关的SoC支持,一部分是添加板级支持。

由于之前的类似项目已经有了完成度较高的工作\footnote{\url{https://github.com/z4yx/u-boot-naivemips/}},本项目计划在其基础上进一步适配。主要的工作有:

\begin{itemize}
    \item 添加串口控制器驱动(包括系统启动早期和后期两个部分)
    \item 添加定时器驱动,准确反映运行时间
    \item 修改设备描述以准确反映SoC片上资源,修复驱动错误(包括SL811 USB模块和DM9000A 网卡驱动)
\end{itemize}