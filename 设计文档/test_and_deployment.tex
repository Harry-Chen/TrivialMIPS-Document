\chapter{测试与部署}

作为软件工程的要求,本项目将进行自动化的集成、测试与部署,共分为硬件,软件与文档三部分。所有的流程都将通过Docker进行,确保是可完整重现的。

\section{硬件部分}
\label{section:test_hardware}

本项目使用的主要硬件设计语言SystemVerilog是一门强大的验证语言,我们将使用其编写testbench来测试硬件模块。主要的testbench也分为三个模块:

\begin{description}
    \item[CPU测试] 本部分用于测试CPU实现指令的正确性。我们对CPU的各条指令都编写了相应的测试程序,同时还对各类可能的冲突现象、异常、TLB的行为编写了对应的测试。测试的过程是通过一个testbench虚拟出外部的总线和RAM并且接入CPU,同时对CPU中WB阶段对寄存器的写请求进行监视。每个测试代码对所有寄存器的写请求都会给出响应期待的结果,同时在运行testbench时会将监测到的真实的写请求和期望的结果进行对比进而确认程序执行的正确性。整个CPU的测试过程是自动化的,无需人工介入观察信号,如果发现真实的运行和期望不符会进行报告。
    \item[外设测试] 本部分用于测试外设控制器的实现正确性(主要是时序)。测试程序作为Master挂接于总线上,其余外设控制器作为Slave正常连接。测试程序向每个外设分别发出不同的读写指令,硬件部分使用对应的仿真模型文件,或者直接观察信号变化,在返回结果不符合预期,或向硬件发出的指令不正确时,终止运行并反馈错误。
    \item[完整测试] 本部分用于模拟整个片上系统的运行。测试时CPU、总线与各个外设控制器正常连接,SRAM、Flash与Bootrom的仿真模型中均配置所需的映像文件。启动测试后,可模拟从加载Bootloader到加载操作系统的全过程。
\end{description}

对于主分支的每一次提交,都需要进行持续集成,包括进行Vivado项目的编译、综合与进行上述的仿真。通常只运行CPU和外设测试部分,在需要时运行完整测试。

\section{软件部分}

在软件方面,本项目计划对编写的所有汇编/C/C++代码,移植的Bootloader、操作系统,以及需要运行的功能测试、性能测试,均编写持续集成脚本,保证每个版本都能进行正确的、可重现的编译。进一步地,借助计原在线平台提供的 ThinPad SDK,可以在上一步硬件部分编译完成bitstream的基础上,将硬件设计与软件一同上传运行,在检验硬件实现正确性的同时,也可以自动化地测试其性能表现。

\section{文档部分}

最后,作为设计文档的结尾,本文档已经实现了计划中的自动集成的功能。对任意文档的每一次修改都能自动编译成对应的版本发布到GitLab,方便开发人员与需求方的查阅。
