\chapter{测试与部署}

作为软件工程课程的需求,为了保证实现的正确性,本项目进行了自动化的集成、测试与部署,共分为硬件、软件与文档三部分。所有的流程都通过Docker进行,确保是可完整复现的。

\section{硬件测试}
\label{section:test_hardware}

本项目使用的主要硬件设计语言SystemVerilog是一门强大的验证语言,我们用它编写编写testbench来测试硬件模块。主要的测试用例分为三类:

\begin{description}
    \item[CPU测试] 本部分用于测试CPU实现指令的正确性。我们对CPU的各条指令都编写了相应的测试程序,同时还对各类可能的冲突现象、异常、TLB的行为编写了对应的测试。测试的过程是通过一个testbench虚拟出外部的总线和RAM并且接入CPU,并对CPU中的访存动作,包括WB阶段对寄存器和的写请求和MEM阶段的内存写请求进行监视。每个测试用例对上述动作都会给出响应期待的结果,同时在运行testbench时会将监测到的真实的写请求和期望的结果进行对比进而确认程序执行的正确性。整个CPU的测试过程是自动化的,通过指定格式的汇编代码即可生成对应的存储文件(\texttt{.mem}文件)、答案文件(\texttt{.ans}文件);我们编写了用于执行上述动作的 SystemVerilog task,可以直接使用 Vivado 对所有用例依次执行测试,得到比较结果,无需人工介入观察信号,如果发现真实的运行和期望不符会进行报告。除此之外,我们还在CPU仿真部分引入了Verilator这一基于C++的编译型仿真工具,用于简单的测试,其性能相比解释型工具有数量级上的增强。
    \item[外设测试] 本部分用于测试外设控制器的实现正确性(主要是时序)。测试程序作为Master挂接于总线上,其余外设控制器作为Slave正常连接。测试程序向每个外设分别发出不同的读写指令,硬件部分或者使用对应的仿真模型文件,或者直接观察信号变化,在返回结果不符合预期,或向硬件发出的指令不正确时,测试程序会终止运行并报告错误。
    \item[完整测试] 本部分用于模拟整个片上系统的运行。测试时CPU、总线与各个外设控制器正常连接,SRAM、Flash与Bootrom的仿真模型中均配置所需的映像文件。启动测试后,可模拟从加载Bootloader到加载操作系统的全过程各个信号的变化。
\end{description}

对于主分支的每一次提交,都需要进行持续集成,包括进行IP核的生成与预综合、Vivado项目的综合与上述的仿真。由于完整仿真速度较慢,通常只运行CPU和外设测试部分。

\section{软件测试}

在软件方面,本项目计划对编写的所有汇编/C/C++代码,移植的Bootloader、操作系统,以及需要运行的功能测试、性能测试,均编写基于GitLab CI的持续集成脚本,保证每个版本都能进行正确的、可重现的编译。进一步地,借助计原在线平台提供的 ThinPad SDK,我们在硬件部分综合完成的bitstream基础上,将硬件设计与软件一同上传运行,在检验硬件实现正确性的同时,也可以自动化地测试其功能、性能表现,以及运行操作系统验证是否正确。

\section{文档生成}

最后,作为设计文档的结尾,本文档实现了计划中的自动集成的功能。对任意文档的每一次修改都会被自动编译成对应的PDF文件发布到GitLab,方便开发人员的查阅。
