\chapter{外设部分}

\section{常用约定}

为了更方便地实现外设与传递信号,我们定义了一些常用的数据类型别名。

\begin{minted}{systemverilog}
typedef logic           Bit_t;
typedef logic [3:0]     Nibble_t;
typedef logic [7:0]     Byte_t;
typedef logic [15:0]    HalfWord_t;
typedef logic [31:0]    Word_t;
typedef logic [63:0]    DoubleWord_t;
typedef logic [3:0]     ByteMask_t;
typedef logic [5:0]     Interrupt_t;

typedef struct packed {
    logic _50M, _11M0592, _10M;
    logic base, base_2x, base_2x_noshift;
    logic rst;
} Clock_t;

`define IRQ_UART     0
`define IRQ_ETHERNET 1
`define IRQ_USB      2

`define EVAL(A) `A
`define REGISTER_IRQ(MODULE, NAME, SOURCE) genvar i; \
    generate \
        for (i = 0; i < $bits(Interrupt_t); i++) begin \
            assign SOURCE[i] = (i == `EVAL(IRQ_``MODULE)) ? NAME : 1'b0; \
        end \
    endgenerate
\end{minted}

其中 \texttt{Interrupt\_t} 用于在外设、总线与CPU间传递中断信号,由于第7个中断被CPU自带的定时中断占据,因此设备可用的仅有6个。使用中断的外设需要使用 \texttt{REGISTER\_IRQ} 这一宏将自己注册到总线的相应位置上。

此外,下列描述中,我们将 CPU 时钟称为“主时钟”,将频率为主时钟的二倍、相位不变的时钟称为“总线时钟”,将总线时钟向前移相 $45\deg$ 后得到的时钟称为外设时钟。各个时钟的相位关系可见图\ref{figure:clock-phase}所示。我们总是假设 CPU 在 \texttt{MAIN\_1} 这一上升沿向总线上发送请求。

\begin{figure}[htbp]
    \centering
    \includegraphics[width=\linewidth]{{timing_bus_crop.pdf}}
    \caption{各个时钟的相位关系}
    \label{figure:clock-phase}
\end{figure}

\section{地址分配与总线}

表\ref{table:address_allocation}列举了数据和地址总线上各个设备的物理地址空间分配。为了使解码和请求的分发更方便,每一个设备都有 16MB 的地址空间,但其只会用到其中的一部分。下面各节将给出各个设备的详细行为描述。

\begin{table}[!htbp]
    \centering
    \begin{tabular}{|c|c|c|c|c|}
    \hline
    \textbf{名称} & \textbf{起始地址} & \textbf{结束地址} & \textbf{有效大小} & \textbf{类型} \\ \hline
    SRAM        & 0x00000000    & 0x007FFFFF    & 8 MB          & 存储          \\ \hline
    Flash       & 0x01000000    & 0x017FFFFF    & 8 MB          & 存储          \\ \hline
    Graphics    & 0x02000000    & 0x02075300    & 480004 B      & 混合          \\ \hline
    UART        & 0x03000000    & 0x03000000    & 2 地址          & 寄存器         \\ \hline
    Timer       & 0x04000000    & 0x04000000    & 2 地址          & 寄存器         \\ \hline
    Ethernet    & 0x05000000    & 0x05000000    & 2 地址          & 寄存器         \\ \hline
    GPIO        & 0x06000000    & 0x06000000    & 3 地址          & 寄存器         \\ \hline
    USB         & 0x07000000    & 0x07000000    & 2 地址          & 寄存器         \\ \hline
    Bootrom     & 0x1FC00000    & 0x1FC03FFF    & 16 KB         & 存储          \\ \hline
    \end{tabular}
    \caption{各外设的物理地址分配}
    \label{table:address_allocation}
\end{table}

CPU向总线传递的所有地址都必须按4对齐,也就是说最后两个二进制位必需是0,否则将会触发一个“地址错误”异常。如果尝试访问超出任何设备“有效大小”之外的地址,或者向声明为只读的地址写数据,后果将是不可预料的。

本项目计划采用 Harvard 架构,因此分指令与数据两条总线。其中指令总线主设备为CPU,从设备为Bootrom与SRAM;数据总线主设备为CPU,从设备为除Bootrom外所有外设。两条总线的工作机制都是一致的,即将CPU的读/写指令通过地址前缀匹配连接到相应设备控制器,并同时将设备输出的连接到CPU。由于此类简单总线不至引起冲突,总线地址分配也互不重合,因此不需要仲裁等进阶功能。

我们使用了下列的 \texttt{interface} 用于定义总线与设备间的连接。其中CPU的master侧连接到总线的slave侧,而所有外设的slave侧分别连接到总线的master侧。传递的信号名称都是自解释的,其中 \texttt{data\_rd\_2} 目前仅用于 Bootrom 与 SRAM 控制器同时读取两条指令,供指令双发射所用。

\begin{minted}{systemverilog}
interface Bus_if (
    input Clock_t clk
);
    Word_t      address;
    Bit_t       read, write;
    Bit_t       stall;
    Word_t      data_rd, data_rd_2, data_wr;
    ByteMask_t  mask;
    Interrupt_t interrupt;

    modport master (
        output address, read, write, data_wr, mask,
        input  stall, data_rd, data_rd_2, interrupt,
        input  clk
    );

    modport slave (
        output stall, data_rd, data_rd_2, interrupt,
        input  address, read, write, data_wr, mask,
        input  clk
    );

endinterface
\end{minted}

\section{存储设备}

SRAM、Flash和Bootrom都是存储设备,其中Bootrom是只读的,SRAM是易失的,Flash则是非易失的。在物理接口上,Flash芯片、SRAM芯片与Bootrom所使用的Xilinx IP核都遵循了所谓的“SRAM接口”,即有片选(CE)、读使能(OE)、写使能(WE)、地址(Address)和数据(Data)信号,其中除数据线为双向(IP核拆分为出入两条),其余均为FPGA需要给出的。三者对读写过程都分别有不同的时序要求,需要根据各自的数据手册正确实现读写时序。此外,Flash芯片还有一些额外的信号,如写保护、重置等,也需要正确处理。

\subsection{Bootrom 控制器}

Bootrom 的实现使用 XilinX 提供的双端口 ROM IP 核,大小为 16 KB,按照字编址,因此传送的地址需要去掉最后i两位。Bootrom 只需要读取,但请求可能来自于两条总线。其两个端口的时钟均为总线时钟。在每个主时钟的周期中,如果只有指令需要读取,我们在 \texttt{PERI\_1} 获得 CPU 的指令地址处理后传递给 bootrom,它在\texttt{BUS\_1} 获得请求后给出数据,这样就能保证CPU在\texttt{MAIN\_2}获得所请求的指令内容;如果同时有来自指令与数据总线的请求,我们先处理数据请求,而在指令总线上拉高 \texttt{stall} 信号,等到\texttt{MAIN\_2}对应的周期再以同样的方式处理指令请求,而后同时送出数据与指令。

\subsection{SRAM 控制器}

SRAM芯片的每个物理地址也是按字编址的。由于需要实现同时两条指令读取的功能,所以我们将存储内容在两片 SRAM 中交错存放,使得连续的两条指令总是存在于两片 SRAM 中。SRAM 控制器使用下列的接口与硬件进行通信:

\begin{minted}{systemverilog}
typedef logic [`SRAM_CHIP_ADDRESS_WIDTH-1 : 0] SramChipAddress_t;

interface Sram_if();
    wire Word_t data;
    SramChipAddress_t address;
    logic[3:0] be_n;
    logic      ce_n, oe_n, we_n;

    modport master(
        output address, be_n, ce_n, oe_n, we_n,
        inout  data
    );

endinterface
\end{minted}

虽然SRAM 本身不需要时钟,但根据手册的描述,在约束文件中,我们在 SRAM 相关信号上附加了的 10 ns 的输入延迟(建立时间)与 8 ns 的输出延迟(保持时间),以保证芯片的时序要求得到满足。其控制器运行在外设时钟上,工作流程为:

在每个主时钟的周期中,如果只有指令需要读/写,我们在 \texttt{PERI\_1} 获得 CPU 的指令后转换为 SRAM 需要的信号送出,在 \texttt{PERI\_2} 取回数据送给 CPU。如果同时有来自指令与数据总线的请求,我们先处理数据的读/写请求,而在指令总线上拉高 \texttt{stall} 信号,等到\texttt{MAIN\_2}对应的周期再以同样的方式处理指令请求,而后同时送出数据与指令。

事实上,由于CPU在给出地址线的请求时没有经过 pipeline,所需的传播延迟较长,所以我们的确无法在同一时间内完成两条总线的请求。因此这一实现是目前最优的方案。

\subsection{Flash 控制器}

Flash的每个物理地址存储的是半字(16 bit),所以在进行读时,控制器内部需要事实上进行两次连续的读操作后将结果拼接起来;在进行写时,Flash芯片只要求半字写,无需进行特殊处理。其与设备的通信接口定义为:

\begin{minted}{systemverilog}
typedef logic [`FLASH_CHIP_ADDRESS_WIDTH-1 : 0] FlashChipAddress_t;

interface Flash_if();
    FlashChipAddress_t address;
    wire HalfWord_t data;
    logic rp_n, vpen, ce_n, oe_n, we_n, byte_n;

    modport master(
        output address, rp_n, vpen, ce_n, oe_n, we_n, byte_n,
        inout  data
    );

endinterface
\end{minted}

Flash在给出读/写标志后,对数据或/和地址保持时间要求为75ns(读)/60ns(写),这意味着一次读写需要多个周期。我们使用一个运行在反相的主时钟上的状态机(即,其每次触发都在主时钟的下降沿)进行读写,如图 \ref{figure:flash-state} 所示。图中的每一个读写状态事实上都是三个状态的组合,其中输入/输出均不发生变化,只是为了满足时序要求。

\begin{figure}[htbp]
    \centering
    \includegraphics[width=0.8\linewidth]{{flash.pdf}}
    \caption{Flash控制器状态机}
    \label{figure:flash-state}
\end{figure}

\section{UART串口}

UART是一种无状态协议,实验板上为其预留了TX与RX两条信号线。本项目将使用开源的串口组件作为底层的收发器,串口控制器共向外暴露两个寄存器地址。第一个寄存器(\texttt{0x03000000})是只读的,最低位指示 CTS (Clear To Send) 信号,表示可以发送数据;次低位是 DR (Data Ready) 信号,表示有未读取的数据。第二个寄存器(\texttt{0x03000004})是可读写的,当 CTS 信号为高,写操作能向TX信号线上发送一个Byte(高位被忽略);当DR信号为高,读操作能够获得一个RX信号线上传来的Byte。

UART控制器模块使用115200的标准波特率,1个停止位,无校验位。读、写两端都应当有足够大的FIFO缓冲区(如4K)来保证读/写不会因为发送过快或一段时间没有取走而产生非预期结果。只要有数据没有被取走,控制器就应当保持拉高一个外部设备中断。

\section{显示控制器}

表中控制器的类型是“混合”,是由于它由 240000 字节的图形缓冲区(framebuffer)和一个处于末尾的(0x2075300)配置寄存器组成。在图形缓冲区中,能存储800*600像素的图像,每个像素占据8比特,格式为\texttt{\{RED[2:0], GREEN[2:0], BLUE[1:0]\}}。控制寄存器是可读写的,用来指示图形缓冲区中第一个像素的偏移量,这可以帮助操作系统渲染终端等画面时方便地实现滚动功能。

本项目将把缓冲区中的内容,以 \texttt{800*600@75Hz} 的画面格式,借助TFP 410芯片产生DVI信号,通过HDMI接口输出。为此,需要借助像素时钟(50MHz),正确地产生符合VGA标准的行、场同步信号,从缓冲区读取并输出相应颜色数据。除此之外,还要正确借助控制寄存器来实现含偏移的渲染。

\section{GPIO控制器}

GPIO控制器包含三个寄存器地址,第一个(\texttt{0x06000000})是只读的拨码开关状态, 第二个(\texttt{0x06000004})和第三个(\texttt{0x06000008})是可读写的,分别代表七段数码管和LED的显示。对应实验板上这三种设备的数量,这些寄存器都只有低16位是有意义的。特别地,当\texttt{0x060004}地址的最高为被置于0时,七段数码管的译码功能被启用,只有最低八位是有效且会被显示的;如果被置于1,则译码停用,七段数码管共有16个笔画(包括两个小数点),恰好能够显示一个16位整数。

\section{定时器}

定时器模块的第一个寄存器(\texttt{0x04000000})包含了一个每1微妙自动递增的整数,并可以被修改为任意值。它可以被用来不受实际CPU频率影响地计量一些指令执行的时间。第二个寄存器(\texttt{0x04000004})包含了CPU主时钟的周期计数。这两个寄存器都会在硬重置时被归零。

\section{以太网与USB控制器}

作为额外的需求,以太网与USB控制器的硬件逻辑实现并不复杂。它们的硬件同样基于SRAM接口,因此读写可以复用部分Flash与SRAM控制器的代码。但需要注意的是二者对时序也另有各自的要求,以太网控制器一般可以在30ns内完成一个请求,而USB控制器则需要较长的等待。二者暴露的第一个寄存器都是可写的地址选择寄存器(SL811使用了最低8位,而DM9000A甚至只是用了最低位),第二个寄存器是可读写的数据寄存器(SL811最低8位有效,DM9000A最低16位有效)。需要特别注意的书,对于同一个地址的多次,DM9000A可能给出不同的结果,这也反映在对其数据寄存器的读取结果上。

这两个部件的操作和通信都较复杂,都需要操作系统中驱动程序的配合才能进行工作。USB控制器与以太网控制器都有硬件的中断信号输入,项目中需要将它们同步后直接连接到CPU的硬件中断端口,以使得操作系统正常处理来自外部硬件的数据和请求。