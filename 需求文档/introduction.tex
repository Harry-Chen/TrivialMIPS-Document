\chapter{概述}

\section{项目背景}

本项目是计算机组成原理与软件工程两门课程的联合实验。项目需求方为计算机组成原理课程,需求方代表为刘卫东老师;项目承担方是“编程是一件很危险的事情”(\textit{ProgrammingCanBeVeryDangerous})小组,组长为计63陈晟祺,成员还包括计64周聿浩和计53姚沛然。

本项目的目标是在新的32位ThinPad实验板上设计并实现基于 MIPS 32 的CPU,并使用实验板上的周边硬件,成为一个片上系统(SOC)。其能够支持标准MIPS 32 Rev 1指令集的一个较完整子集和 MIPS 32 Rev 2指令集的部分功能,并能够运行 uCore 操作系统。在此基础上,我们还将尝试移植 uCore Plus 操作系统。

\section{项目概览}

本项目计划设计和实现的部分主要包括:CPU、外设、Bootloader、操作系统移植。项目使用的硬件语言为 SystemVerilog 2005。下面为各个部分的概览。

\subsection{CPU}

CPU 的设计包含指令集、流水线结构(微架构)、内存管理单元、异常处理机制、协处理器以及其他增强功能。

\begin{description}

    \item[指令集] 本项目的CPU实现的指令是 MIPS 32 Rev 1 指令集的一个较完整子集,包括了所有的算术逻辑指令、控制流指令和大部分特权指令(不包括与缓存有关的),覆盖了 uCore 操作系统需要的所有47条指令。MIPS 32 Rev 2 中的部分指令(如 CP0 中的 ebase 寄存器)由于被操作系统需要,也包含在设计中。
    \item[流水线结构] 本项目计划实现经典的MIPS五级流水线结构,即分为取指、译码、执行、访存、回写阶段,每个阶段在CPU内部使用一个时钟周期。为此,需要解决一系列数据和控制流上的冲突、竞争。
    \item[内存管理单元] 本项目计划实现内存管理单元(MMU)以进行从虚拟地址到物理地址的映射,本项目的内存划分遵循 MIPS 32 标准,将使用转换检测缓冲区(TLB)以加速页表的查询,并对所有外设实现内存映射IO(MMIO)。
    \item[异常处理机制] 本项目计划完整支持 MIPS 32 Rev 1的异常和中断机制,正确处理同步和异步异常,支持硬件和软件中断,并实现精确异常。
    \item[协处理器] 本项目将实现 MIPS 32 Rev 1中为CP0处理器规定的几乎所有指令和寄存器,以正确运行操作系统。
    \item[增强功能] 如果在规定时间内完成了上述需求,本项目还可能实现一系列性能与功能的增强,如CP1浮点协处理器、微架构调整(指令多重发射)等。

\end{description}

\subsection{外设}

\subsection{Bootloader}

\subsection{操作系统移植}